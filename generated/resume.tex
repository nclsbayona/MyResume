% Copyright 2013 Christophe-Marie Duquesne <chmd@chmd.fr>
% Copyright 2014 Mark Szepieniec <http://github.com/mszep>
% 
% ConText style for making a resume with pandoc. Inspired by moderncv.
% 
% This CSS document is delivered to you under the CC BY-SA 3.0 License.
% https://creativecommons.org/licenses/by-sa/3.0/deed.en_US

\startmode[*mkii]
  \enableregime[utf-8]  
  \setupcolors[state=start]
\stopmode

\setupcolor[hex]
\definecolor[titlegrey][h=757575]
\definecolor[sectioncolor][h=397249]
\definecolor[rulecolor][h=9cb770]

% Enable hyperlinks
\setupinteraction[state=start, color=sectioncolor]

\setuppapersize [A4][A4]
\setuplayout    [width=middle, height=middle,
                 backspace=20mm, cutspace=0mm,
                 topspace=10mm, bottomspace=20mm,
                 header=0mm, footer=0mm]

%\setuppagenumbering[location={footer,center}]

\setupbodyfont[11pt, helvetica]

\setupwhitespace[medium]

\setupblackrules[width=31mm, color=rulecolor]

\setuphead[chapter]      [style=\tfd]
\setuphead[section]      [style=\tfd\bf, color=titlegrey, align=middle]
\setuphead[subsection]   [style=\tfb\bf, color=sectioncolor, align=right,
                          before={\leavevmode\blackrule\hspace}]
\setuphead[subsubsection][style=\bf]

\setuphead[chapter, section, subsection, subsubsection][number=no]

%\setupdescriptions[width=10mm]

\definedescription
  [description]
  [headstyle=bold, style=normal,
   location=hanging, width=18mm, distance=14mm, margin=0cm]

\setupitemize[autointro, packed]    % prevent orphan list intro
\setupitemize[indentnext=no]

\defineitemgroup[enumerate]
\setupenumerate[each][fit][itemalign=left,distance=.5em,style={\feature[+][default:tnum]}]

\setupfloat[figure][default={here,nonumber}]
\setupfloat[table][default={here,nonumber}]

\setuptables[textwidth=max, HL=none]
\setupxtable[frame=off,option={stretch,width}]

\setupthinrules[width=15em] % width of horizontal rules

\setupdelimitedtext
  [blockquote]
  [before={\setupalign[middle]},
   indentnext=no,
  ]

\usesymbols[fontawesome]
\def\fa#1{\symbol[fontawesome][#1]}


\starttext

\section[title={Nicolas Bayona},reference={nicolas-bayona}]

\startblockquote
Systems engineering student passionate about technological stuff and the
use of technology to improve people's lives. You can check my
\useURL[url1][https://linkedin.com/][][\fa{linkedin}LinkedIn]\from[url1]
\stopblockquote

\thinrule

\subsection[title={Education},reference={education}]

2019-2023 (expected):\crlf
{\bf BSc, Systems engineering}; Pontifical Xaverian University (Bogota,
Colombia)

\startitemize[packed]
\item
  Final project title: Still not sure :C
\stopitemize

\subsection[title={Experience},reference={experience}]

{\bf Volunteer Work Experience:}

{\em {\bf Academic mentor at Pontifical Xaverian University}}:

\startitemize
\item
  As an academic mentor at the university I was in charge of:

  \startitemize[packed]
  \item
    Welcoming new coming students to a new stage in their lives.
  \item
    Helping the new coming students in their adaptation process.
  \item
    Tell students about my experience as an student so far, my greatest
    challenges and important lessons I learnt throughout my life.
  \item
    Act as a leader to new coming students, so that they felt they were
    accompanied in their adaptation processes.
  \stopitemize
\stopitemize

\subsection[title={Technical
Experience},reference={technical-experience}]

Most of my technical work is in my GitHub profile, you can visit my
profile at
\useURL[url2][https://github.com/nclsbayona][][\fa{github}GitHub]\from[url2]

{\bf My Cool Side Projects:}\crlf
Usually I practice my skills via some helpful project that I tend to
mantain over time, I think it's really good to automate repetitive tasks
and jobs a bot or some type of automation software can do on its own.
Examples of this are my

\startitemize[packed]
\item
  These items can also contain lists, but you need to mind the
  indentation levels in the markdown source.
\item
  Second item.
\stopitemize

{\bf Open Source:} * {\bf {\em Project 1:}} Here

\startitemize
\item
  {\bf {\em Project 2:}} Description
\item
  {\bf {\em Project 3:}} We both
\stopitemize

{\bf Programming Languages:}\crlf
Throughout my life, I've worked in many projects and therefore, I've
adquired a lot of experience in various programming languages. Next,
I'll name a few and some relevant stuff I learnt from them.

\startitemize
\item
  {\bf {\em C++:}}\crlf
  The first programming language I learnt was C++, the most valuable
  thing I think I learnt from my C++ learning was programming logic. I
  think that C++ is a really great language because it acts as a basis
  for other languages and many tools that have been working flawlessly
  since it first appeared almost 40 years ago.

  \startitemize[packed]
  \item
    {\em Advantages:}
    \startitemize[packed]
    \item
      C++ is really efficient in terms of memory consumption, execution
      time, and comparatively in energy consumption thanks to being a
      compiled language.
    \item
      C++ allows the use of not only classes but c-like structs too,
      which gives the language a lot of possibilities and extensibility
      too.
    \stopitemize
  \item
    {\em Disadvantages:}
    \startitemize[packed]
    \item
      C++ is a complex language, not only because of its syntax but also
      because of other stuff like pointers and security, and therefore,
      it requires a lot of time to implement algorithms in C++.
    \item
      Due to C++ being a compiled language, executables are strictly
      attached to an specific architecture, an executable file might
      present problems when trying to execute the program in an
      architecture different from the one it was compiled in the first
      place.
    \item
      There exist libraries that don't offer cross-platform versions,
      and that might not work in an environment different from the one
      it was developed in first place, this allows programs to work only
      in a specific machine.
    \stopitemize
  \stopitemize
\item
  {\bf {\em Python:}}\crlf
  After learning C++, I decided to learn Python, because its reputation
  was great, it simplified the development process as its syntax was
  similar to natural language and included really powerful mathematical
  tools that helped to perform tasks that might involve mathematical
  stuff.

  \startitemize[packed]
  \item
    {\em Advantages:}
    \startitemize[packed]
    \item
      Python has a lot of cool modules that in a lot of situations.
    \item
      C++ allows the use of not only classes but c-like structs too,
      which gives the language a lot of possibilities and extensibility
      too.
    \stopitemize
  \item
    {\em Disadvantages:}
    \startitemize[packed]
    \item
      C++ is a complex language, not only because of its syntax but also
      because of other stuff like pointers and security, and therefore,
      it requires a lot of time to implement algorithms in C++.
    \item
      Due to C++ being a compiled language, executables are strictly
      attached to an specific architecture, an executable file might
      present problems when trying to execute the program in an
      architecture different from the one it was compiled in the first
      place.
    \item
      There exist libraries that don't offer cross-platform versions,
      and that might not work in an environment different from the one
      it was developed in first place, this allows programs to work only
      in a specific machine.
    \stopitemize
  \stopitemize
\item
  {\bf {\em Java:}}\crlf
  We both know this one's pushing it.
\item
  {\bf {\em Other relevant information:}}\crlf
  Also, I have basic knowledge of other programming languages like
  {\em {\bf TypeScript}, {\bf JavaScript}, {\bf C}, {\bf Dart} and
  {\bf Swift}}, as well as markup languages such as {\em {\bf YAML},
  {\bf HTML}, , {\bf Markdown}, {\bf XML} and {\bf JSON}}, and other
  tech-related tools like {\em {\bf Kubernetes - Kubectl}, {\bf Docker -
  Podman}, {\bf Github Actions} and {\bf AWS}}.
\stopitemize

\subsection[title={Aditional
information},reference={aditional-information}]

\startitemize
\item
  Human Languages:

  \startitemize[packed]
  \item
    Spanish (Native speaker)
  \item
    English (Fully-conversational)
  \stopitemize
\stopitemize

\thinrule

\useURL[url3][mailto:bayona.n@javeriana.edu.co][][bayona.n@javeriana.edu.co]\from[url3]
• Bogota, Colombia\crlf
Reach me on
\useURL[url4][https://t.me/nclsbayona][][\fa{telegram}]\from[url4]Telegram,
my username is nclsbayona

\thinrule

\stoptext
