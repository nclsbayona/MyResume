% Copyright 2013 Christophe-Marie Duquesne <chmd@chmd.fr>
% Copyright 2014 Mark Szepieniec <http://github.com/mszep>
% 
% ConText style for making a resume with pandoc. Inspired by moderncv.
% 
% This CSS document is delivered to you under the CC BY-SA 3.0 License.
% https://creativecommons.org/licenses/by-sa/3.0/deed.en_US

\startmode[*mkii]
  \enableregime[utf-8]  
  \setupcolors[state=start]
\stopmode

\setupcolor[hex]
\definecolor[titlegrey][h=757575]
\definecolor[sectioncolor][h=397249]
\definecolor[rulecolor][h=9cb770]

% Enable hyperlinks
\setupinteraction[state=start, color=sectioncolor]

\setuppapersize [A4][A4]
\setuplayout    [width=middle, height=middle,
                 backspace=20mm, cutspace=0mm,
                 topspace=10mm, bottomspace=20mm,
                 header=0mm, footer=0mm]

%\setuppagenumbering[location={footer,center}]

\setupbodyfont[11pt, helvetica]

\setupwhitespace[medium]

\setupblackrules[width=31mm, color=rulecolor]

\setuphead[chapter]      [style=\tfd]
\setuphead[section]      [style=\tfd\bf, color=titlegrey, align=middle]
\setuphead[subsection]   [style=\tfb\bf, color=sectioncolor, align=right,
                          before={\leavevmode\blackrule\hspace}]
\setuphead[subsubsection][style=\bf]

\setuphead[chapter, section, subsection, subsubsection][number=no]

%\setupdescriptions[width=10mm]

\definedescription
  [description]
  [headstyle=bold, style=normal,
   location=hanging, width=18mm, distance=14mm, margin=0cm]

\setupitemize[autointro, packed]    % prevent orphan list intro
\setupitemize[indentnext=no]

\defineitemgroup[enumerate]
\setupenumerate[each][fit][itemalign=left,distance=.5em,style={\feature[+][default:tnum]}]

\setupfloat[figure][default={here,nonumber}]
\setupfloat[table][default={here,nonumber}]

\setuptables[textwidth=max, HL=none]
\setupxtable[frame=off,option={stretch,width}]

\setupthinrules[width=15em] % width of horizontal rules

\setupdelimitedtext
  [blockquote]
  [before={\setupalign[middle]},
   indentnext=no,
  ]


\starttext

\section[title={Nicolas Bayona},reference={nicolas-bayona}]

\thinrule

\startblockquote
Systems engineering student passionate about technological stuff and the
use of technology to improve people's lives.
\stopblockquote

\thinrule

\subsection[title={Education},reference={education}]

2019-2023 (expected):\crlf
{\bf BSc, Systems engineering}; Pontifical Xaverian University (Bogota,
Colombia)

\starttyping
* Final project title: Still not sure :C
\stoptyping

\subsection[title={Experience},reference={experience}]

{\bf Volunteer Work Experience:}

{\em {\bf Academic mentor at Pontifical Xaverian University}}:

\startitemize
\item
  As an academic mentor at the university I was in charge of:

  \startitemize[packed]
  \item
    Welcoming new coming students to a new stage in their lives.
  \item
    Helping the new coming students in their adaptation process.
  \item
    Tell students about my experience as an student so far, my greatest
    challenges and important lessons I learnt throughout my life.
  \item
    Act as a leader to new coming students, so that they felt they were
    accompanied in their adaptation processes.
  \stopitemize
\stopitemize

\subsection[title={Technical
Experience},reference={technical-experience}]

Most of my technical work is in my GitHub profile, you can visit my
profile at \useURL[url1][https://github.com/nclsbayona]\from[url1]

{\bf My Cool Side Project:}\crlf
For items which don't have a clear time ordering, a definition list can
be used to have named items.

\starttyping
* These items can also contain lists, but you need to mind the
  indentation levels in the markdown source.
* Second item.
\stoptyping

{\bf Open Source:}\crlf
List open source contributions here, perhaps placing emphasis on the
project names, for example the {\bf Linux Kernel}, where you implemented
multithreading over a long weekend, or {\bf node.js} (with
\useURL[url2][http://nodejs.org][][link]\from[url2]) which was actually
totally your idea\ldots{}

{\bf Programming Languages:}\crlf
{\bf {\em C++:}} Here, we have an itemization, where we only want to add
descriptions to the first few items, but still want to mention some
others together at the end. A format that works well here is a
description list where the first few items have their first word
emphasized, and the last item contains the final few emphasized terms.
Notice the reasonably nice page break in the pdf version, which wouldn't
happen if we generated the pdf via html.

{\bf {\em Python:}} Description of your experience with second-lang,
perhaps again including a {[}link{]} {[}ref{]}, this time placing the
url reference elsewhere in the document to reduce clutter (see source
file).

{\bf {\em Java:}} We both know this one's pushing it.

Basic knowledge of other programming languages like
{\em {\bf TypeScript}, {\bf JavaScript}, {\bf C}, {\bf Dart} and
{\bf Swift}}.\crlf
As well as markup languages such as {\em {\bf YAML}, {\bf HTML},
{\bf XML} and {\bf JSON}}.\crlf
And tools like {\em {\bf Kubernetes - Kubectl}, {\bf Docker - Podman},
{\bf Github Actions}, {\bf AWS}}.

\subsection[title={Aditional
information},reference={aditional-information}]

\startitemize
\item
  Human Languages:

  \startitemize[packed]
  \item
    Spanish (Native speaker)
  \item
    English (Fully-conversational)
  \stopitemize
\stopitemize

\thinrule

\useURL[url3][mailto:bayona.n@javeriana.edu.co][][bayona.n@javeriana.edu.co]\from[url3]
• Bogota, Colombia\crlf
Reach me on \useURL[url4][https://t.me/nclsbayona][][Telegram: My
username is nclsbayona]\from[url4]

\stoptext
